\documentclass[paper=a4, fontsize=11pt]{scrartcl} % A4 paper and 11pt font size

\usepackage{listings}
\lstset{ %
  backgroundcolor=\color{white},   % choose the background color; you must add \usepackage{color} or \usepackage{xcolor}
  basicstyle=\footnotesize,        % the size of the fonts that are used for the code
  breakatwhitespace=false,         % sets if automatic breaks should only happen at whitespace
  breaklines=true,                 % sets automatic line breaking
  captionpos=b,                    % sets the caption-position to bottom
  commentstyle=\color{black},    % comment style
  deletekeywords={...},            % if you want to delete keywords from the given language
  escapeinside={\%*}{*)},          % if you want to add LaTeX within your code
  extendedchars=\true,              % lets you use non-ASCII characters; for 8-bits encodings only, does not work with UTF-8
  frame=single,                    % adds a frame around the code
  keepspaces=true,                 % keeps spaces in text, useful for keeping indentation of code (possibly needs columns=flexible)
  keywordstyle=\color{black},       % keyword style
  language=Octave,                 % the language of the code
  morekeywords={*,...},            % if you want to add more keywords to the set
%  numbers=left,                    % where to put the line-numbers; possible values are (none, left, right)
%  numbersep=5pt,                   % how far the line-numbers are from the code
%  numberstyle=\tiny\color{blue}, % the style that is used for the line-numbers
  rulecolor=\color{black},         % if not set, the frame-color may be changed on line-breaks within not-black text (e.g. comments (green here))
  showspaces=false,                % show spaces everywhere adding particular underscores; it overrides 'showstringspaces'
  showstringspaces=false,          % underline spaces within strings only
  showtabs=false,                  % show tabs within strings adding particular underscores
  stepnumber=2,                    % the step between two line-numbers. If it's 1, each line will be numbered
  stringstyle=\color{blue},     % string literal style
  tabsize=2,                       % sets default tabsize to 2 spaces
  title=\lstname,                   % show the filename of files included with \lstinputlisting; also try caption instead of title
  inputencoding=ansinew
}
\lstset{literate=%
{æ}{{\ae}}1
{å}{{\aa}}1
{ø}{{\o}}1
{Æ}{{\AE}}1
{Å}{{\AA}}1
{Ø}{{\O}}1
{Ö}{{\"O}}1
{á}{{\'a}}1
}

\usepackage[titletoc]{appendix}
\usepackage{subcaption}
\usepackage{tkz-graph}
\renewcommand*{\EdgeLineWidth}{0.15pt}
\usepackage{listings} 
\usepackage[T1]{fontenc} % Use 8-bit encoding that has 256 glyphs
\usepackage{fourier} % Use the Adobe Utopia font for the document - comment this line to return to the LaTeX default
\usepackage[english]{babel} % English language/hyphenation
\usepackage{amsmath,amsfonts,amsthm} % Math packages
\usepackage{lipsum} % Used for inserting dummy 'Lorem ipsum' text into the template
\usepackage{graphicx}
\usepackage{sectsty} % Allows customizing section commands
\allsectionsfont{\centering \normalfont\scshape} % Make all sections centered, the default font and small caps
\usepackage[utf8]{inputenc} 
\usepackage{fancyhdr} % Custom headers and footers
\pagestyle{fancyplain} % Makes all pages in the document conform to the custom headers and footers
\fancyhead{} % No page header - if you want one, create it in the same way as the footers below
\fancyfoot[L]{} % Empty left footer
\fancyfoot[C]{} % Empty center footer
\fancyfoot[R]{\thepage} % Page numbering for right footer
\renewcommand{\headrulewidth}{0pt} % Remove header underlines
\renewcommand{\footrulewidth}{0pt} % Remove footer underlines
\setlength{\headheight}{13.6pt} % Customize the height of the header

\numberwithin{equation}{section} % Number equations within sections (i.e. 1.1, 1.2, 2.1, 2.2 instead of 1, 2, 3, 4)
\numberwithin{figure}{section} % Number figures within sections (i.e. 1.1, 1.2, 2.1, 2.2 instead of 1, 2, 3, 4)
\numberwithin{table}{section} % Number tables within sections (i.e. 1.1, 1.2, 2.1, 2.2 instead of 1, 2, 3, 4)

\setlength\parindent{0pt} % Removes all indentation from paragraphs - comment this line for an assignment with lots of text

%----------------------------------------------------------------------------------------
%	TITLE SECTION
%----------------------------------------------------------------------------------------

\newcommand{\horrule}[1]{\rule{\linewidth}{#1}} % Create horizontal rule command with 1 argument of height

\title{	
\normalfont \normalsize 
\textsc{Københavns Universitet, Datalogisk Institut} \\ [25pt] % Your university, school and/or department name(s)
\horrule{0.5pt} \\[0.4cm] % Thin top horizontal rule
\huge A Compiler for the FASTO Language\\ % The assignment title
\horrule{2pt} \\[0.5cm] % Thick bottom horizontal rule
}

\author{Allan Nielsen, Christian Nielsen, Troels Kamp Leskes} % Your name

\date{\normalsize\today} % Today's date or a custom date

\begin{document}


\maketitle % Print the title
\tableofcontents

\pagebreak


\section{Multiplication, division, boolean operators and literals}

	Implementing multiplication and division was a simple matter, when having the already implemented code for addition and subtraction to look at. They served as a great way of getting to know the fasto compiler, and how things operate.
	
	Negation is implemeted using the Mips.SUB instruction, where we pass the original argument to the operator, and subtracts this from zero.

	So the instruction looks like:
	\begin{verbatim}
		Mips.SUB(place, "0", t1),
	\end{verbatim}

	where t1 register that holds the argument $x$ for $\textasciitilde x$, and place is the register in which we place the result.

	Not was more complicated than the previous ones, given that this requires more than one instruction to execute. However, the pattern learned here, proved to be useful for implementing and and or as well.

	\begin{verbatim}
		[ Mips.LI (place,"0")
		, Mips.BNE (b,"0",falseLabel)
		, Mips.LI (place,"1")
		, Mips.LABEL falseLabel ]
    \end{verbatim}
    
    Place is the register in which we want to store our result. We start by putting 0 into place, we then check if our argument b, actually is 0. Since Mips.BNE branches if its arguments are not equal, we will jump to falseLabel if and only if our argument b is 1, thus ending with a 0 in the place register, given we never execute Mips.LI(place, "1").\\\\
    
For boolean literals we have added "true" and "false" as cases in the lexer, just before matching on pretty much everything and checking in keywords, else these would be caught there. Another approach could have been adding them to the list of keywords, but we chose not to do this.\\
\begin{lstlisting}
| "true"|"false"      { case Bool.fromString (getLexeme lexbuf) of
                       NONE   => lexerError lexbuf "Bad bool"
                     | SOME b => Parser.BOOLEAN (b, getPos lexbuf) }
\end{lstlisting}
These string are then passed on to the parser where they then are cast to booleans.\\
\verb"| BOOLEAN       { Constant ( BoolVal (#1 $1), #2 $1) }"
\\
Furthermore these are being passed on as constants and evaluated in the code-generator:
\begin{lstlisting}
| Constant (BoolVal b, pos) =>
    if(b) then
      [Mips.LI(place, "1")]
    else
      [Mips.LI(place, "0")]
\end{lstlisting}
 
In our associativity we have the following code:

\begin{lstlisting}
21  %nonassoc ifprec letprec
22  %left DEQ LTH
23  %left OR 
24  %left AND
25  %nonassoc NOT
26  %left PLUS MINUS
27  %left TIMES DIV
28  %nonassoc NEG
\end{lstlisting}
Because of this hopefully we are getting the right bindings in our expressions. In the associations we made the \%prec ifprec addition to the \textit{if then else} and also the \textit{not} statements. Because of this we are able to evaluate an expression like \textit{not 4 == 2} as \textit{not (4 == 2)} even though the \textit{not} operator have a tighter associative binding than \textit{==}. We have been testing the associativity in two test cases \textit{assoc.fo} and \textit{assoc2.fo} to ensure that the bindings are right for all of our operators.   
 
\section{Filter and scan}

	\subsection{Filter}
	For filter we expect a function with a return type of bool, and some type of array. We make sure this is the case in our typechecker, raise errors if we get a non-array argument or a non-bool function. If everything checks up, we among other things pass the type of the array to the code generator.
	\\\\
	Given that filter takes some function and an array as input, and runs this function over the array, we notice that filter and map are similar in ways, so we based the code for filter on the code for map. With the difference being, that we do not want to store some computed value, instead we want to store the original value of the element we might be iterating on, only of the input function returns true.
	\\\\
	We acheived this by making changes to the load part, so that we now store the actual value (instead of the computed one), in a register.
	
	Since our input and output arrays will be of different sizes, we created a new counter \textit{j\_reg}, which gets incremented only when the input function on an element from the array, returns true.
	\\\\
	Between the load and save sequence of map, we made a \textit{check\_fun} instruction set;
	
	\begin{lstlisting}
	val check_fun = [ Mips.BEQ (res_reg, "0", loop_beg)
	                          , Mips.ADDI (j_reg, j_reg, "1") ]
	\end{lstlisting}
	where \textit{res\_reg} is the register in which we place the result of the function call on the input function. We branch to the loop beginning, if the result is false.

	\subsubsection{Testing}
	We used several tests for filter, making sure we use different types, as well as regular and lambda functions.
	
	A test with a regular function, on integers.
	\begin{lstlisting}
	fun bool great(int n) = 5 < n

	fun int writeInt(int n) = write(n)

	fun [int] main() =
  		let a = filter(great, {6, 9, 8, 2}) in
  		map(writeInt, a)
	\end{lstlisting}

	Similar test to the above, but with a lambda function, on integers.
	\begin{lstlisting}
	fun int writeInt(int n) = write(n)

	fun [int] main() =
		  let a = filter(fn bool (int x) => 5 < x, {6, 9, 8, 2, ~5}) in
		  map(writeInt, a)
	\end{lstlisting}

	A test with a lambda function on bools.
	\begin{lstlisting}
	fun bool writeBool(bool s) = write(s)

	fun [bool] main() =
		let a = {true, false, false, true} in
		let b = filter(fn bool (bool x) => not x, a) in
		map(writeBool, b)
	\end{lstlisting}

	All three tests perform as expected.
\subsection{Scan}
To implement \verb"scan", we wanted to reuse code from map and reduce, by combining the iterator from map and the accumulator from reduce in the code-generator.\\
In the typechecker, we pretty much reused the whole thing from reduce, adding that it should return an array of the return-type of the given function argument.\\\\
Below is the changed snippet for type-checking for \verb"scan":
\begin{lstlisting}
...
              then (Array elem_type,
                       Out.Scan (f', n_dec, arr_dec, elem_type, pos))
...
\end{lstlisting}

The idea to implement this, was to accumulate the values using the given function into a placeholder and put that into the return-array as the function progress. Next thing was how to get the initial value into the first index of the return-array, which we found intuitively had to be done before the whole loop-process had begun.\\
The initializing code ended up like this:
\begin{lstlisting}
...
val init_regs = [ Mips.ADDI (addr_reg, place, "4") (* point to the next word *)
                , Mips.SW (acc_reg, addr_reg, "0")
                , Mips.ADDI (addr_reg, addr_reg, "4") (* point to next word*)
                , Mips.MOVE(i_reg, "0") (* initialize iterator*)
                , Mips.ADDI (arr_reg, arr_reg, "4") (* Look at first element in input array *)]
...
\end{lstlisting}

For the looping itself, we split it into 5 parts, for the sole reason we found it easier to debug this way.\\
A header, that would check if the condition is satisfied:\\
\begin{lstlisting}
...
            val loop_head =
                [ Mips.LABEL(loop_beg) 
                , Mips.SUB(tmp_reg, i_reg, out_size_reg) (* make statement*)
                , Mips.BGEZ(tmp_reg, loop_end) ] (* if statement is equal to zero, jump to end *)
...
\end{lstlisting}	

An upperbody, that loads the value of the next element in the input array to be evaluated:\\
\begin{lstlisting}
...
val load_value =
  case getElemSize elem_type of
    One =>  [ Mips.LB   (elem_reg, arr_reg, "0")
            , Mips.ADDI (arr_reg, arr_reg, "1") ]
  | Four => [ Mips.LW   (elem_reg, arr_reg, "0")
            , Mips.ADDI (arr_reg, arr_reg, "4") ]
...
\end{lstlisting}

Next is where the magic happens, the application of the function on the accumulator/placeholder along with the newly loaded element. This is, as mentioned, put into the placeholder for next iteration:\\
\begin{lstlisting}
...
val apply_code =
    applyFunArg(farg, [acc_reg, elem_reg], vtable, acc_reg, pos)
...
\end{lstlisting}

Afterwards we want to save this into the new array and count up the address register for next iteration.
\begin{lstlisting}
...
val save_value =
  case getElemSize elem_type of
    One => [ Mips.SB (acc_reg, addr_reg, "0")
           , Mips.ADDI (addr_reg, addr_reg, "1")] 
 | Four => [ Mips.SW (acc_reg, addr_reg, "0") 
           , Mips.ADDI (addr_reg, addr_reg, "4")]
...
\end{lstlisting}

Last, but not least we increment the iterator and jump to the header for the next iteration:\\
\begin{lstlisting}
...
val loop_foot = [ Mips.ADDI(i_reg, i_reg, "1")
                , Mips.J loop_beg
                , Mips.LABEL loop_end ]
\end{lstlisting}


\subsubsection{Testing}
When testing this, we found it does not work properly with bools. A simple test as this:
\begin{lstlisting}
fun bool and(bool a, bool b) = a && b
fun bool or(bool a, bool b) = a || b

fun bool writeBool(bool n) = write(n)

fun [bool] main() =
  let a = {true, true, true, true, true, true, true, true} in
  let b = scan(or, true, a) in
  map(writeBool, b)
\end{lstlisting}
 Yields the output \verb"TrueFalseFalseFalseTrue...True" depending how big you make the array with true, but the 2nd to 4th always stays false. We believe it might have something to do in our code-generator some byte offset not being set porperly, but is somehow corrected when the iteration has run a few times.\\\\
For integers it works fine, as the test below yields what is expected.
\begin{lstlisting}
fun int multi(int a, int b) = a * b

fun int writeInt(int n) = write(n)

fun [int] main() =
  let a = {1, 2, 3, 4} in
  let b = map(fn int (int x) => x*2, a) in 
  let c = scan(multi, 1, b) in
  map(writeInt, c)
\end{lstlisting}
\section{$\lambda$-expressions in soacs}

To implement this we wanted to use as much from regular functions as possible, so we started going through each step, starting from the lexer, discussing how we could use already used code for function, to our advantage.\\\\
The lexing and parser part was quite simple, as we know $\lambda$-functions always begin with \verb"fn" and the rest looks like a regular function, so we just pass everything on like a regular function. In the parser we put the whole thing under FunArgs, since $\lambda$-functions only are defined as being used in \verb"map, filter, scan and reduce".\\
Below is the \verb"FunArg"-type shown
\begin{lstlisting}
FunArg : ID { FunName (#1 $1) }
    | FN Type LPAR Params RPAR ARROW Exp
      { Lambda ( $2, $4, $7, $1 ) }
    | FN Type LPAR RPAR ARROW Exp
      { Lambda ( $2, [], $6, $1 ) }
;
\end{lstlisting}

As shown, these arguments are passed on to the \verb"Lambda"-function in the typeChecker.\\
In the typeChecker, we create a \verb"FunDec" using a dummy-name, as suggested. This is then passed to \verb"checkFunWithVtable" where the function is typechecked, then the parameters are typechecked and its passed to the code-generator.\\
Below is the code for typechecking a \verb"Lambda"-function shown:
\begin{lstlisting}
| checkFunArg (In.Lambda (ret_type, params, body, funpos)
                , vtab, ftab, pos) = 
      let val (Out.FunDec ( name, _, params, body', pos)) = checkFunWithVtable (In.FunDec ("Lambda", ret_type, params, body, pos), vtab, ftab, funpos)
        val arg_types = map (fn (Param (_, ty)) => ty) params
      in  
      (Out.Lambda(ret_type, params, body', funpos), ret_type, arg_types)
      end
\end{lstlisting}

In the code-generator we need to bind the parameters with the arguments in the SymTab and do so recursively using a local defined function \verb"bindVars", that is only available in that scope.\\
The base-cases is checking if one of the two arrays is empty while the other is not and throw and error if that is the case. Otherwise we want to return to updated vtable, once both the array of arguments and the array of parameters both are depleted simultaneously.\\
With the new vtable we then compute the expression for the \verb"Lambda"-function.\\
Below is the code-generation for \verb"Lambda" shown
\begin{lstlisting}
  | applyFunArg (Lambda(ret_type, params, body', funpos), args, vtable, place, pos) : Mips.Prog = 
      let 
        fun bindVars ([], [], vtable) = vtable
          | bindVars([], args, vtable) = raise Error("stop det pjat", pos)
          | bindVars(params, [], vtable) = raise Error("stop det pjat stadigvæk", pos)
          | bindVars(Param (name, paramtype)::params, arg::args, vtable) = SymTab.bind name arg (bindVars(params, args, vtable))
      val newVtable = bindVars(params, args, vtable)
      val code1 = compileExp body' newVtable place
      in
        code1
      end
\end{lstlisting}

To test this, we simply created a test-case for each of the 4 SOACS that can use $\lambda$ -functions.\\

Below is the test-case for $\lambda$-function in map SOAC. The tests for the other SOACs is found in the tests folder for our project.
\begin{lstlisting}
fun int writeInt(int x) = write(x)

fun [int] writeIntArr([int] x) = map(writeInt, x)

fun int main() =
   let N = read(int) in
   let z = iota(N)   in
   let w = writeIntArr(map(fn int (int x) => x + 2, z)) in
   let nl = write("\n") in
   writeInt(reduce(op+, 0, w))
\end{lstlisting}

\section{Copy propagation and constant folding}

	We have implemented \textit{copy propagation and constant foldning} in our compiler, though it does not handle shadowing. Furthermore have all the features from task 1.\\
Below is the test-case for Times  explained. This will indicate that both the cases work and it is able to actually optimize by propagating and fold correspondingly.\\

When we want to fold expressions, we can do so, by predicting what the result is going to be of a given expression and return this instead.\\
We thought through for each expression, what base-cases there could be and what we then should return.\\

Below is shown our code for the \textit{Times}-expression. 
\begin{lstlisting}
| Times  (e1, e2, pos) =>
  let val e1' = copyConstPropFoldExp vtable e1
      val e2' = copyConstPropFoldExp vtable e2
  in case (e1', e2') of
         (Constant (IntVal x, _), Constant (IntVal y, _)) =>
         Constant (IntVal (x*y), pos)
       | (Constant (IntVal 0, _), _) =>
         e1'
       | (_, Constant (IntVal 0, _)) =>
         e2'
       | (Constant (IntVal 1, _), _) =>
         e2'
       | (_, Constant (IntVal 1, _)) =>
         e1'
       | _ =>
         Times (e1', e2', pos)
  end
\end{lstlisting}

We start by optimizing the two subexpressions. Then we check if any of the two expressions evaluates to 0 or 1. If one of them is 0, we return 0. Also, if one of them is 1 then it will be the other expression that is returned, no matter what it states. If none of the above applies, then we want to compute the expression.\\\\
Below is the code for testing \textit{Times}. 
\begin{lstlisting}
fun int main() = 
  let a = 5 in
  let b = a in
  let c = b in
   write(b + c * 0)
\end{lstlisting}
If our propagation is correct, all the variables should evaluate to 5, due to \textit{a} having that constant assigned.\\
In the \textit{write-}statement we have some expressions which also can be optimized. Since we multiply by 0, that whole expression will evaluate to 0. Then we plus \textit{b} with 0 and end up with \textit{b} as the result, and we end up writing \textit{b}.\\
In the end, our program will look like this when optimized using copy propagation and constant folding.
\begin{lstlisting}
fun int main() =  
    let a = 5 in    
    let b = 5 in    
    let c = 5 in
    write(5)
\end{lstlisting}

\pagebreak

\section*{Appendix}

\appendix
The following appendix shows the code we have added in the compiler phases Lexer.lex, Parser.grm, TypeChecker.sml, CodeGen.sml, Interpreter.sml and CopyConstPropFold.sml\\\\

\chapter{Lexer.lex}

\begin{lstlisting}
40	  | "fn"           => Parser.FN pos 
42	  | "not"          => Parser.NOT pos

66	  | "true"|"false"      { case Bool.fromString (getLexeme lexbuf) of 
67	                               NONE   => lexerError lexbuf "Bad bool" 
68	                             | SOME b => Parser.BOOLEAN (b, getPos lexbuf) }

87	  | `*`                 { Parser.TIMES (getPos lexbuf) } 
88	  | `/`                 { Parser.DIV (getPos lexbuf) } 
89	  | `~`                 { Parser.NEG (getPos lexbuf) }
91	  | "=>"                { Parser.ARROW (getPos lexbuf) } 
94	  | "||"                { Parser.OR (getPos lexbuf) } 
95	  | "&&"                { Parser.AND (getPos lexbuf) } 
\end{lstlisting}

\pagebreak
\chapter{Parser.grm}
\begin{lstlisting}
10  %token <(int*int)> IF THEN ELSE LET IN INT BOOL CHAR EOF
11  %token <string*(int*int)> ID STRINGLIT
12  %token <int*(int*int)> NUM
13  %token <char*(int*int)> CHARLIT
14  %token <bool*(int*int)> BOOLEAN
15  %token <(int*int)> PLUS MINUS DEQ EQ LTH NEG NOT ARROW
16  %token <(int*int)> TIMES DIV AND NOT OR LPAR RPAR LBRACKET RBRACKET LCURLY RCURLY
17  %token <(int*int)> COMMA
18  %token <(int*int)> FUN FN IOTA REPLICATE MAP REDUCE FILTER SCAN READ WRITE 
19  %token <(int*int)> OP

21  %nonassoc ifprec letprec
22  %left DEQ LTH
23  %left OR 
24  %left AND
25  %nonassoc NOT
26  %left PLUS MINUS
27  %left TIMES DIV
28  %nonassoc NEG

84          | Exp TIMES Exp  
85                          { Times($1, $3, $2) }
86          | Exp DIV Exp  
87                          { Divide($1, $3, $2) }
88          | NEG Exp  
89                          { Negate($2, $1) }
90          | Exp AND Exp { And($1, $3, $2) }
91          | Exp OR Exp 
92                          { Or($1, $3, $2) }

117	        | SCAN LPAR FunArg COMMA Exp COMMA Exp RPAR 
118	                        { Scan ($3, $5, $7, (), $1)} 
119	        | FILTER LPAR FunArg COMMA Exp RPAR 
120	                        { Filter ($3, $5, (), $1) } 

126	        | NOT Exp %prec ifprec { Not ( $2, $1 ) }   

133	FunArg : ID { FunName (#1 $1) } 
134	    | FN Type LPAR Params RPAR ARROW Exp 
135	      { Lambda ( $2, $4, $7, $1 ) } 
136	    | FN Type LPAR RPAR ARROW Exp 
137	      { Lambda ( $2, [], $6, $1 ) } 
138	;

\end{lstlisting}

\pagebreak
\chapter{TypeChecker.sml}

\begin{lstlisting}
255     | In.Times (e1, e2, pos)	 
256	      => let val (_, e1_dec, e2_dec) = checkBinOp ftab vtab (pos, Int, e1, e2) 	 	
257	          in (Int,  	 	
258	            Out.Times(e1_dec, e2_dec, pos))  	 	
259	          end   	 	
260	 	 	
261	    | In.Divide (e1, e2, pos) 	 	
262	      => let val (_, e1_dec, e2_dec) = checkBinOp ftab vtab (pos, Int, e1, e2) 	 	
263	          in (Int,  	 	
264	            Out.Divide(e1_dec, e2_dec, pos))  	 	
265	          end 	 	
266	 	 	
267	    | In.Negate (e1, pos) 	 	
268	      => let val (t1, e1') = checkExp ftab vtab e1 	 	
269	         in (Int, Out.Negate(e1', pos)) 	 	
270	         end 	 	
271	     	 	
272	    | In.Not (e1, pos) 	 	
273	      => let val (t1, e1') = checkExp ftab vtab e1 	 	
274	         in (Bool, Out.Not(e1', pos)) 	 	
275	         end   	 	
276	 	 	
277	    | In.Or (e1, e2, pos) 	 	
278	      => let val (t1, e1') = checkExp ftab vtab e1 	 	
279	             val (t2, e2') = checkExp ftab vtab e2 	 	
280	             in 	 	
281	                if(t1 = Bool andalso t2 = Bool) then 	 	
282	                  (Bool, Out.Or(e1', e2', pos)) 	 	
283	                else 	 	
284	                  raise Error("Type Error: Non-boolean arguments given to ||", pos) 	 	
285	          end 	 	
286	 	 	
287	    | In.And (e1, e2, pos) 	 	
288	      => let val (t1, e1') = checkExp ftab vtab e1 	 	
289	             val (t2, e2') = checkExp ftab vtab e2 	 	
290	             in 	 	
291	                if(t1 = Bool andalso t2 = Bool) then 	 	
292	                  (Bool, Out.And(e1', e2', pos)) 	 	
293	                else 	 	
294	                  raise Error("Type Error: Non-boolean arguments given to &&", pos) 	 	
295	          end   	 	
296	     	 	
298	    | In.Scan (f, n_exp, arr_exp, _, pos)	256	 
299	      => let val (n_type, n_dec) = checkExp ftab vtab n_exp 	 	
300	             val (arr_type, arr_dec) = checkExp ftab vtab arr_exp 	 	
301	             val elem_type = 	 	
302	               case arr_type of 	 	
303	                  Array t => t 	 	
304	                | other => raise Error ("Scan: argument is not an array", pos) 	 	
305	             val (f', f_arg_type) =  	 	
306	               case checkFunArg (f, vtab, ftab, pos) of 	 	
307	                   (f', res, [a1, a2]) => 	 	
308	                   if a1 = a2 andalso a2 = res 	 	
309	                   then (f', res) 	 	
310	                   else raise Error 	 	
311	                          ("Scan: incompatible function type of " 	 	
312	                           ^ In.ppFunArg 0 f ^": " ^ showFunType ([a1, a2], res), pos) 	 	
313	                 | (_, res, args) => 	 	
314	                   raise Error ("Scan: incompatible function type of " 	 	
315	                                ^ In.ppFunArg 0 f ^ ": " ^ showFunType (args, res), pos) 	 	
316	             fun err (s, t) = 	 	
317	                 Error ("Scan: unexpected " ^ s ^ " type " ^ ppType t ^ 	 	
318	                        ", expected " ^ ppType f_arg_type, pos) 	 	
319	         in if elem_type = f_arg_type 	 	
320	            then if elem_type = n_type 	 	
321	                 then (Array elem_type, 	 	
322	                       Out.Scan (f', n_dec, arr_dec, elem_type, pos)) 	 	
323	                 else raise (err ("neutral element", n_type)) 	 	
324	            else raise err ("array element", elem_type) 	 	
325	         end 	 	

329	    | In.Filter (f, arr_exp, _, pos)	260	 
330	      => let val (arr_type, arr_exp_dec) = checkExp ftab vtab arr_exp 	 	
331	              	 	
332	              val elem_type = 	 	
333	                case arr_type of 	 	
334	                  Array t => t 	 	
335	                | other => raise Error ("Filter: argument is not an array", pos) 	 	
336	 	 	
337	              val (f', f_res_type, f_arg_type) = 	 	
338	                case checkFunArg (f, vtab, ftab, pos) of 	 	
339	                    (f', Bool, [a1]) => (f', Bool, a1) 	 	
340	                  | (_,  res, args) => 	 	
341	                   raise Error ("Filter: incompatible function type of " 	 	
342	                                ^ In.ppFunArg 0 f ^ ":" ^ showFunType (args, res), pos) 	 	
343	                     	 	
344	              in (arr_type, 	 	
345	                      Out.Filter (f', arr_exp_dec, elem_type, pos)) 	 	
346	             end
 	
362	    | checkFunArg (In.Lambda (ret_type, params, body, funpos)	276	 
363	                , vtab, ftab, pos) =  	 	
364	      let val (Out.FunDec ( name, _, params, body', pos)) = checkFunWithVtable (In.FunDec ("Lambda", ret_type, params, body, pos), vtab, ftab, funpos) 	 	
365	      val arg_types = map (fn (Param (_, ty)) => ty) params 	 	
366	      in   	 	
367	      (Out.Lambda(ret_type, params, body', funpos), ret_type, arg_types) 	 	
368	      end 	
\end{lstlisting}

\pagebreak
\chapter{CodeGen.sml}
\begin{lstlisting}
614	    | Constant (BoolVal b, pos) => 
615	        if(b) then 
616	          [Mips.LI(place, "1")] 
617	        else 
618	          [Mips.LI(place, "0")] 

624     | Times (e1, e2, pos) => 
625	        let val t1 = newName "times_L" 
626	            val t2 = newName "times_R" 
627	            val code1 = compileExp e1 vtable t1 
628	            val code2 = compileExp e2 vtable t2 
629	        in code1 @ code2 @ [Mips.MUL (place,t1,t2)] 
630	        end 
631	    | Divide (e1, e2, pos) => 
632	        let val t1 = newName "div_L" 
633	            val t2 = newName "div_R" 
634	            val code1 = compileExp e1 vtable t1 
635	            val code2 = compileExp e2 vtable t2 
636	        in code1 @ code2 @ [Mips.DIV (place,t1,t2)] 
637	        end 
638	 
639	    | Negate (e1, pos) => 
640	        let val t1 = newName "negateVal" 
641	            val code1 = compileExp e1 vtable t1 
642	        in code1 @ [ Mips.SUB(place, "0", t1) ] 
643	        end 

645	    | Not (b_exp, pos) => 
646	        let val b = "boolean" 
647	            val code1 = compileExp b_exp vtable b 
648	            val falseLabel = newName "false" 
649	        in code1 @ 
650	            [ Mips.LI (place,"0") 
651	            , Mips.BNE (b,"0",falseLabel) 
652	            , Mips.LI (place,"1") 
653	            , Mips.LABEL falseLabel ] 
654	        end 
655	     
656	    | Or (e1, e2, pos) => 
657	        let val trueLabel = newName "trueLabel" 
658	            val falseLabel = newName "falseLabel" 
659	            val endLabel = newName "endLabel" 
660	            val code1 = compileCond e1 vtable trueLabel falseLabel 
661	            val code2 = compileCond e2 vtable trueLabel endLabel 
662	        in  [Mips.LI(place, "0")] @  
663	              code1 @  
664	              [Mips.LABEL falseLabel] @  
665	              code2 @  
666	              [Mips.LABEL trueLabel, Mips.LI(place, "1"), Mips.LABEL endLabel] 
667	        end 
668
669	    | And (e1, e2, pos) => 
670	        let val trueLabel = newName "trueLabel" 
671	            val falseLabel = newName "falseLabel" 
672	            val endLabel = newName "endLabel" 
673	            val code1 = compileCond e1 vtable trueLabel falseLabel 
674	            val code2 = compileCond e2 vtable endLabel falseLabel 
675	        in  [Mips.ADD(place, "0", "1")] @  
676	              code1 @  
677	             [Mips.LABEL trueLabel] @  
678	              code2 @  
679	             [Mips.LABEL falseLabel, Mips.ADD(place, "0", "0"), Mips.LABEL endLabel] 
680	        end 	          

688	    (* Scan(f, e, [a1, a2, ..., an]) = [e, f(e, a1), f(f(e, a1), a2), ...] *)
689	  | Scan (farg, acc_exp, arr_exp, elem_type, pos) => 
690	        let val in_size_reg = newName "in_size_reg" (* size of input array *) 
691	            val out_size_reg = newName "out_size_reg" (* size of ouput array *) 
692	            val acc_reg = newName "acc_reg" (* last computed value for output *) 
693	            val i_reg = newName "i_reg" (* Iterator register *) 
694	            val addr_reg = newName "addr_reg" (* address of element in output array *)  
695	            val arr_reg = newName "arr_reg" 
696	            val elem_reg = newName "elem_reg" 
697	 
698	            val arr_code = compileExp arr_exp vtable arr_reg 
699	            val acc_code = compileExp acc_exp vtable acc_reg 
700	             
701	            val get_size = [ Mips.LW (in_size_reg, arr_reg, "0") (* Loads array-size into in_size_reg *) 
702	                           , Mips.ADDI(out_size_reg, in_size_reg, "1") (* Puts size into out_size_reg *) 
703	                           ]  
704	 
705	            (* Initiate registers. 
706	               Put address of place into addr_reg, so we return proper addresses. 
707	               Increment in_size_reg by 1 to determine out_size_reg, since output array is 
708	               1 element longer. *) 
709	            val init_regs = [ Mips.ADDI (addr_reg, place, "4") (* point to the next word *) 
710	                            , Mips.SW (acc_reg, addr_reg, "0") 
711	                            , Mips.ADDI (addr_reg, addr_reg, "4") (* point to next word*) 
712	                            , Mips.MOVE(i_reg, "0") (* initialize iterator*) 
713	                            , Mips.ADDI (arr_reg, arr_reg, "4") (* Look at first element in input array*) 
714	                            ] 
715	  
716	  
717	            val loop_beg = newName "loop_beg" 
718	            val loop_end = newName "loop_end" 
719	            val tmp_reg = newName "tmp_reg" 
720	  
721	            (* while i_reg < in_size_reg*) 
722	            val loop_head = 
723	                [ Mips.LABEL(loop_beg)  
724	                , Mips.SUB(tmp_reg, i_reg, out_size_reg) (* make statement*) 
725	                , Mips.BGEZ(tmp_reg, loop_end) ] (* if statement is equal to zero, jump to end *) 
726	            (* loads next value in input array into tmp_reg *) 
727	            val load_value = 
728	                case getElemSize elem_type of 
729	                    One =>  [ Mips.LB   (elem_reg, arr_reg, "0") 
730	                            , Mips.ADDI (arr_reg, arr_reg, "1") ] 
731	                  | Four => [ Mips.LW   (elem_reg, arr_reg, "0") 
732	                            , Mips.ADDI (arr_reg, arr_reg, "4") ] 
733	                
734	            val apply_code = 
735	                applyFunArg(farg, [acc_reg, elem_reg], vtable, acc_reg, pos) 
736	  
737	            (* save the current accumulated value to a register for later computations, 
738	               and to memory for later output *) 
739	            val save_value = 
740	                case getElemSize elem_type of 
741	                    One => [ Mips.SB (acc_reg, addr_reg, "0") 
742	                           , Mips.ADDI (addr_reg, addr_reg, "1")] 
743	                  | Four => [ Mips.SW (acc_reg, addr_reg, "0")  
744	                            , Mips.ADDI (addr_reg, addr_reg, "4")] 
745	  
746	  
747	  
748	            (* increments i_reg *) 
749	            val loop_foot = 
750	                [ Mips.ADDI(i_reg, i_reg, "1") 
751	                , Mips.J loop_beg 
752	                , Mips.LABEL loop_end ] 
753	            
754	        in [Mips.LABEL "Det_her_er_starten"]  
755	           @ arr_code 
756	           @[Mips.LABEL "array_kode"]  
757	           @ acc_code 
758	           @[Mips.LABEL "akku_kode"]  
759	           @ get_size 
760	           @[Mips.LABEL "Stoerrelsen_på_dyret"]  
761	           @ dynalloc (out_size_reg, place, elem_type) 
762	           @[Mips.LABEL "init_regz"]  
763	           @ init_regs 
764	           @[Mips.LABEL "starten_af_loopet"]  
765	           @ loop_head 
766	           @[Mips.LABEL "midten_af_loopet"]  
767	           @ load_value 
768	           @[Mips.LABEL "mere_midte"]  
769	           @ apply_code 
770	           @[Mips.LABEL "gem_den_akku"]  
771	           @ save_value 
772	           @[Mips.LABEL "foden_af_loopet"]  
773	           @ loop_foot 
774	           @ [Mips.LABEL "Det_her_er_slutningen_SCAN"] 
775	        end 

783	    (* filter(f(), acc, [a1,a2]) = [f(acc, a1), f(acc, a2)] *) 
784	    | Filter (farg, arr_exp, elem_type, pos) => 
785	        let val size_reg = newName "size_reg" (* size of input/output array *) 
786	            val arr_reg  = newName "arr_reg" (* address of input array *) 
787	            val elem_reg = newName "elem_reg" (* address of single element *) 
788	            val res_reg = newName "res_reg" 
789	            val val_reg = newName "val_reg" 
790	            val arr_code = compileExp arr_exp vtable arr_reg 
791	 
792	            val get_size = [ Mips.LW (size_reg, arr_reg, "0") ] (* *) 
793	 
794	            val addr_reg = newName "addr_reg" (* address of element in new array *) 
795	            val i_reg = newName "i_reg" 
796	            val j_reg = newName "j_reg" 
797	            val init_regs = [ Mips.ADDI (addr_reg, place, "4") (*point to the next word, so we don't overwrite the size*) 
798	                            , Mips.MOVE (i_reg, "0") 
799	                            , Mips.LI (j_reg, "0") 
800	                            , Mips.ADDI (elem_reg, arr_reg, "4") ] 
801	 
802	            val loop_beg = newName "loop_beg" 
803	            val loop_end = newName "loop_end" 
804	            val tmp_reg = newName "tmp_reg" 
805	            val loop_header = [ Mips.LABEL (loop_beg) 
806	                              , Mips.SUB (tmp_reg, i_reg, size_reg) 
807	                              , Mips.BGEZ (tmp_reg, loop_end) 
808	                              , Mips.ADDI (i_reg, i_reg, "1") ] 
809	 
810	            (* map is 'arr[i] = f(old_arr[i])'. *) 
811	            val loop_load = 
812	                case getElemSize elem_type of 
813	                    One  => Mips.LB(val_reg, elem_reg, "0") 
814	                              :: applyFunArg(farg, [val_reg], vtable, res_reg, pos) 
815	                            @ [ Mips.ADDI(elem_reg, elem_reg, "1") ] 
816	                  | Four => Mips.LW(val_reg, elem_reg, "0") 
817	                              :: applyFunArg(farg, [val_reg], vtable, res_reg, pos) 
818	                            @ [ Mips.ADDI(elem_reg, elem_reg, "4") ] 
819	             
820	            val check_fun = [ Mips.BEQ (res_reg, "0", loop_beg) 
821	                            , Mips.ADDI (j_reg, j_reg, "1") ] 
822	 
823	            val loop_save = 
824	                case getElemSize elem_type of 
825	                    One => [ Mips.SB (val_reg, addr_reg, "0")] 
826	                  | Four => [ Mips.SW (val_reg, addr_reg, "0") ] 
827	 
828	            val loop_footer = 
829	                [ Mips.ADDI (addr_reg, addr_reg, 
830	                             makeConst (elemSizeToInt (getElemSize elem_type))) 
831	                , Mips.J loop_beg 
832	                , Mips.LABEL loop_end 
833	                , Mips.SW (j_reg, place, "0") 
834	                ] 
835	        in [Mips.LABEL "array_kode"]  
836	           @ arr_code (* make arr_reg point to input array*) 
837	           @[Mips.LABEL "measuring"]  
838	           @ get_size (* gets the size of the input array *) 
839	           @[Mips.LABEL "allocating_da_mem"]  
840	           @ dynalloc (size_reg, place, elem_type) (* return place, which is an address where the ouput array is going to be *) 
841	           @[Mips.LABEL "initializing"]  
842	           @ init_regs 
843	           @[Mips.LABEL "loopin"]  
844	           @ loop_header 
845	           @ loop_load 
846	           @ check_fun 
847	           @ [Mips.LABEL "saving"]  
848	           @ loop_save 
849	           @ loop_footer 
850	        end 

868      | applyFunArg (Lambda(ret_type, params, body', funpos), args, vtable, place, pos) : Mips.Prog =  
868	      let  
869	        fun bindVars ([], [], vtable) = vtable 
870	          | bindVars([], args, vtable) = raise Error("stop det pjat", pos) 
871	          | bindVars(params, [], vtable) = raise Error("stop det pjat stadigvæk", pos) 
872	          | bindVars(Param (name, paramtype)::params, arg::args, vtable) = SymTab.bind name arg (bindVars(params, args, vtable)) 
873	      val newVtable = bindVars(params, args, vtable) 
874	      val code1 = compileExp body' newVtable place 
875	      in 
876	        code1 
877	      end 
\end{lstlisting}

\pagebreak
\chapter{Interpreter.sml}
\begin{lstlisting}
412	  | evalExp ( Times(e1, e2, pos), vtab, ftab ) = 
413	        let val res1   = evalExp(e1, vtab, ftab) 
414	            val res2   = evalExp(e2, vtab, ftab) 
415	        in  evalBinopNum(op *, res1, res2, pos) 
416	        end 
417	 
418	  | evalExp ( Divide(e1, e2, pos), vtab, ftab ) = 
419	        let val res1   = evalExp(e1, vtab, ftab) 
420	            val res2   = evalExp(e2, vtab, ftab) 
421	        in  evalBinopNum(op div, res1, res2, pos) 
422	        end 
423	 
424	  | evalExp ( Negate(e1, pos), vtab, ftab ) = 
425	        let val res1 = evalExp(e1, vtab, ftab) 
426	            val res2 = evalExp(Constant(IntVal 0, pos), vtab, ftab) 
427	        in  evalBinopNum(op +, res1, res2, pos) 
428	        end 
429	 
430	  | evalExp ( Not(e1, pos), vtab, ftab ) = 
431	        let val res1 = evalExp(e1, vtab, ftab) 
432	        in  case res1 of 
433	              BoolVal true => BoolVal false 
434	            | BoolVal false => BoolVal true 
435	            | _ => raise Error("Input is not of type bool", pos) 
436	        end 
437	 
438	  | evalExp ( And(e1, e2, pos), vtab, ftab ) = 
439	        let val res1 = evalExp(e1, vtab, ftab) 
440	            val res2 = evalExp(e2, vtab, ftab) 
441	        in  case res1 of 
442	              BoolVal false => res1 
443	            | BoolVal true => (case res2 of 
444	                                BoolVal _ => res2 
445	                                | _ => raise Error("Second argument is not of type bool", pos)) 
446	            | _ => raise Error("First argument is not of type bool", pos) 
447	       
448	        end 
449	 
450	  | evalExp ( Or(e1, e2, pos), vtab, ftab ) = 
451	        let val res1 = evalExp(e1, vtab, ftab) 
452	            val res2 = evalExp(e2, vtab, ftab) 
453	        in  case res1 of 
454	              BoolVal true => res1 
455	            | BoolVal false => (case res2 of 
456	                                BoolVal _ => res2 
457	                                | _ => raise Error("Second argument is not of type bool", pos)) 
458	            | _ => raise Error("First argument is not of type bool", pos) 
459	       
460	        end

517	  | evalFunArg (Lambda(ret_type, params, exp, pos), vtab, ftab, callpos) = 
518	    let 
519	      val fundec = FunDec ("Lambda", ret_type, params, exp, pos) 
520	    in 
521	     (fn aargs => callFunWithVtable(fundec, aargs, vtab, ftab, callpos), ret_type)  
522	    end 
\end{lstlisting}

\pagebreak
\chapter{CopyConstPropFold.sml}
\begin{lstlisting}
16   fun copyConstPropFoldExp vtable e =
17      case e of
18          Constant x => Constant x
19        | StringLit x => StringLit x
20        | ArrayLit (es, t, pos) =>
21          ArrayLit (map (copyConstPropFoldExp vtable) es, t, pos)
22        | Var (name, pos) =>
23          (* TODO TASK 4: This case currently does nothing.
24        
25           You must perform a lookup in the symbol table and if you find
26           a Propagatee, return either a new Var or Constant node. *)
27          let
28            val value = SymTab.lookup name vtable
29          in
30            case value of
31              SOME (VarProp v)   => copyConstPropFoldExp vtable (Var(v, pos))
32            | SOME (ConstProp v) => copyConstPropFoldExp vtable (Constant(v, pos))
33            | NONE               => Var(name, pos)
34          end

102 		let val e' = copyConstPropFoldExp vtable e
103        in
104          let
105            val vtable' = 
106              case e' of
107                (Var (vname, p))     => (SymTab.bind name (VarProp(vname)) vtable)
108              | (Constant (vval, p)) => (SymTab.bind name (ConstProp(vval)) vtable)
109              | _ => vtable
110          in
111           Let (Dec (name, e', decpos), copyConstPropFoldExp vtable' body, pos)
112          end
113        end
\end{lstlisting}

\end{document}