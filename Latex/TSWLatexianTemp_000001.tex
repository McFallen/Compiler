\documentclass[paper=a4, fontsize=11pt]{scrartcl} % A4 paper and 11pt font size

\usepackage{subcaption}
\usepackage{tkz-graph}
\renewcommand*{\EdgeLineWidth}{0.15pt}
\usepackage{listings} 
\usepackage[T1]{fontenc} % Use 8-bit encoding that has 256 glyphs
\usepackage{fourier} % Use the Adobe Utopia font for the document - comment this line to return to the LaTeX default
\usepackage[english]{babel} % English language/hyphenation
\usepackage{amsmath,amsfonts,amsthm} % Math packages
\usepackage{lipsum} % Used for inserting dummy 'Lorem ipsum' text into the template
\usepackage{graphicx}
\usepackage{sectsty} % Allows customizing section commands
\allsectionsfont{\centering \normalfont\scshape} % Make all sections centered, the default font and small caps
\usepackage[utf8]{inputenc} 
\usepackage{fancyhdr} % Custom headers and footers
\pagestyle{fancyplain} % Makes all pages in the document conform to the custom headers and footers
\fancyhead{} % No page header - if you want one, create it in the same way as the footers below
\fancyfoot[L]{} % Empty left footer
\fancyfoot[C]{} % Empty center footer
\fancyfoot[R]{\thepage} % Page numbering for right footer
\renewcommand{\headrulewidth}{0pt} % Remove header underlines
\renewcommand{\footrulewidth}{0pt} % Remove footer underlines
\setlength{\headheight}{13.6pt} % Customize the height of the header

\numberwithin{equation}{section} % Number equations within sections (i.e. 1.1, 1.2, 2.1, 2.2 instead of 1, 2, 3, 4)
\numberwithin{figure}{section} % Number figures within sections (i.e. 1.1, 1.2, 2.1, 2.2 instead of 1, 2, 3, 4)
\numberwithin{table}{section} % Number tables within sections (i.e. 1.1, 1.2, 2.1, 2.2 instead of 1, 2, 3, 4)

\setlength\parindent{0pt} % Removes all indentation from paragraphs - comment this line for an assignment with lots of text

%----------------------------------------------------------------------------------------
%	TITLE SECTION
%----------------------------------------------------------------------------------------

\newcommand{\horrule}[1]{\rule{\linewidth}{#1}} % Create horizontal rule command with 1 argument of height

\title{	
\normalfont \normalsize 
\textsc{Københavns Universitet, Datalogisk Institut} \\ [25pt] % Your university, school and/or department name(s)
\horrule{0.5pt} \\[0.4cm] % Thin top horizontal rule
\huge A Compiler for the FASTO Language\\ % The assignment title
\horrule{2pt} \\[0.5cm] % Thick bottom horizontal rule
}

\author{Allan Nielsen, Christian Nielsen, Troels Kamp Leskes} % Your name

\date{\normalsize\today} % Today's date or a custom date

\begin{document}


\maketitle % Print the title
\tableofcontents

\pagebreak


\section{Multiplication, division, boolean operators and literals}

	Implementing multiplication and division was a simple matter, when having the already implemented code for addition and subtraction to look at. They served as a great way of getting to know the fasto compiler, and how things operate.
	
	Negation is implemeted using the Mips.SUB instruction, where we pass the original argument to the operator, and subtracts this from zero.

	So the instruction looks like:
	\begin{verbatim}
		Mips.SUB(place, "0", t1),
	\end{verbatim}

	where t1 register that holds the argument $x$ for $\textasciitilde x$, and place is the register in which we place the result.

	Not was more complicated than the previous ones, given that this requires more than one instruction to execute. However, the pattern learned here, proved to be useful for implementing and and or as well.

	\begin{verbatim}
		[ Mips.LI (place,"0")
		, Mips.BNE (b,"0",falseLabel)
		, Mips.LI (place,"1")
		, Mips.LABEL falseLabel ]
    \end{verbatim}
    
    Place is the register in which we want to store our result. We start by putting 0 into place, we then check if our argument b, actually is 0. Since Mips.BNE branches if its arguments are not equal, we will jump to falseLabel if and only if our argument b is 1, thus ending with a 0 in the place register, given we never execute Mips.LI(place, "1").

	\Large{HAR INGEN IDE OM BOOL LITS}

\section{Filter and scan}

\section{$\lambda$-expressions in soacs}

\section{Copy propagation and constant folding}


\end{document}